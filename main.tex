%\documentclass[11pt, oneside]{book}
\documentclass[11pt, twoside]{book}  %importante cambiar si es libro
\usepackage[utf8]{inputenc}
\usepackage[T1]{fontenc}


%%%%%%%%%%%%%%%%%%%%%%%%%%%%%%%%%%%%%%%%%%%%%%%%%%%%%%%%%
%   Paquetes y librerías iniciales para un documento,   
%           Graficos, simbolos, tablas, bibliografía
%%%%%%%%%%%%%%%%%%%%%%%%%%%%%%%%%%%%%%%%%%%%%%%%%%%%%%%%%
%---------------------------------------------------------------------------
% Esto es para que el LaTeX sepa que el texto está en español:
\usepackage[spanish,es-noshorthands,es-tabla]{babel}
\selectlanguage{spanish}
%---------------------------------------------------------------------------
% para uso en modo matemático
\usepackage{amssymb}
\usepackage{amsthm}
\usepackage{amsmath}

%---------------------------------------------------------------------------
% Para las imagenes 

\usepackage[dvips]{graphicx}
\graphicspath{{Figuras/}} % se fija el camino para las figuras
\usepackage{rotating}  %para poder rotar las figuras
\usepackage{wrapfig}

%---------------------------------------------------------------------------
% PIES DE FOTOS Y TABLAS 

\usepackage{float} % para objetos flotantes
\usepackage{caption} % para los pies de fotos o tablas
\usepackage{subcaption}
\captionsetup{font=small,labelfont=bf} % para tamaño y tipo fuentes pie de foto
\usepackage{multirow} % para las tablas unir columnas
\usepackage{multicol} %para escribir en multiples columnas

% Referencias y biblio

\usepackage{pdfpages}
%\usepackage[pdftex,breaklinks=true,hidelinks]{hyperref} % para referencias/indice en pdf

\usepackage[
breaklinks=true,colorlinks=true,
linkcolor=blue,urlcolor=blue,citecolor=blue,% PDF VIEW
linkcolor=black,urlcolor=black,citecolor=black,% PRINT
bookmarks=true,bookmarksopenlevel=2]{hyperref}

\usepackage{emptypage}
\usepackage{ragged2e} %Justificacion

%\usepackage[toc,title,page]{appendix}

%Para bibliografía
\usepackage[numbers,square]{natbib}
%\bibliographystyle{apalike}
\bibliographystyle{IEEEtran}

%%%%%%%%% HAbituales estilos
%plainnat
%%abbrvnat
%unsrtnat  %Items in bibliography sorted in order cited)
%IEEEtran

%

%%%%%%%%%%%%%%%%%%%% Formato de la ual %%%%%%%%%%%%%%%%%%%%%%%%%%%%%%%%%


%\usepackage{UAL}
\usepackage[print,doble]{UAL}
\usepackage{codigo}

%%%%%% PAQUETES INCORPORADOS POR EL USUARIO

\usepackage{longtable}   %Tablas largas
\usepackage{verbatim}
\usepackage{textcomp} %para completar los signos válidos
\usepackage{eurosym} %simbolo del euro [\euro]
\usepackage[bottom]{footmisc}



%%%%%%%%%%%%%%%%%%%%%%%%%%%%%%%%%%%%%%%%%%%%%%%%%%%
% Datos identificativos del TFG                            %
%%%%%%%%%%%%%%%%%%%%%%%%%%%%%%%%%%%%%%%%%%%%%%%%%%%

\author{Isabel María del Águila Cano}
\titulo{Libro de estilo para la  memoria del TFG de informática de la UAL }
\subTitulo{DEPARTAMENTO DE
INFORMÁTICA\\  ÁREA DE LENGUAJES Y SISTEMAS INFORMÁTICOS}
\estudios{Grado en Informática }
\universidad{UNIVERSIDAD DE ALMERÍA}
\director{Fulanito de tal} 
\direct{Menganito de cual}
\curso {2020/2021}


\pagestyle{cab}

\begin{document}
\frontmatter

\portada

\begin{dedicatoria}
    A continuación, la dedicatoria está dirigida a los integrantes del grupo de trabajo que han hecho posible este proyecto.
\end{dedicatoria}

\begin{ListaCapturas}
\end{ListaCapturas}


%%%%%%%%%%%%%%%%%%%%%%%%%%%%%%%%%%%%%%%%%%%%%%%%%%%%%%%%%%%%%%%%%%%%%%%%%%%%
%%%%%%%%%%%%%%% ESTO ES PARA LA TABLA DE CONTENIDOS (INDICE)%%%%%%%%%%%%%%%%


\setcounter{secnumdepth}{3} %niveles en capitulos hasta (1.1.1)
\setcounter{tocdepth}{3} %niveles en indice hasta (1.1)


\addtocontents{toc}{~\hfill\small{Página}\par} %añade "pagina" a tabla de contanidos
\addtocontents{toc}{\vspace{2pt} \hrule \vspace{5mm} \par}

%\renewcommand\contentsname{Índice de contenidos}
\tableofcontents
\listoffigures % indice de figuras
\listoftables % indice de tablas
\lstlistoflistings % indice de listados
\listof{Capturas}{Capturas} %Lista de Capturas

%---------------------------------------------------------------------------
% comienzo de relacion abreviaturas y/o acrónimos
%---------------------------------------------------------------------------


%\addcontentsline{toc}{section}{ABREVIATURAS}
\clearpage
\vspace{0.2cm}
\section*{ABREVIATURAS}

\begin{tabular}{ l   |    l  }
	

&\\
   ESI & Escuela Superior de ingeniería \\
   InSo & Ingeniería del Software \\
    SBSE& Search based software engineering\\
   TFG & Trabajo fin de grado\\
   
      UAL & Universidad de Almería \\
      &\\

\end{tabular}


%\addcontentsline{toc}{section}{ABREVIATURAS}



% Borra el bloque si sólo queremos resumen en contraportada
\addcontentsline{toc}{chapter}{Resumen y Abstract}

\chapter*{Resumen y Abstract}

\input{resumentexto.tex}


%se suelen utilizar archivos separados, por capitulos para agilizar la compilación en las versiones intermedias
\mainmatter
\marcagua

\input{contenido1}

\input{contenido2}


\addcontentsline{toc}{section}{BIBLIOGRAFÍA }
\bibliography{referencias}

\appendix
%\clearpage

\input{apendices}
%\input{pmbok}

\clearpage
\backmatter
\input {contra}

\end{document}

